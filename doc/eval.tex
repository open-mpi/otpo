In this section, an MCA parameter that we mentioned in section two will be
evaluated. The parameter {\tt btl\_openib\_recieve\_queues} will be
evaluated. We saw that this parameter is a combination of several recieve
queue, where each queue in turn takes other parameters. 

\subsection{Configuration File and Results}
For OTPO to handle this kind of parameter, this parameter should be specified
as an aggregate of other virtual parameters. In our test, we decided to use
two recieve queues, one Per-peer queue and one shared receive queue. The
following specification was used for the MCA parameter:

\begin{itemize}
\item {\tt btl\_openib\_rq\_1\_type}: P
\item {\tt btl\_openib\_rq\_1\_size}: 65536 $->$ 262144 (increment * 2)
\item {\tt btl\_openib\_rq\_1\_num}: 1 $->$ 256 (increment * 2)
\item {\tt btl\_openib\_rq\_1\_low\_wat}: 1 $->$ 64 (increment * 2)
\item {\tt btl\_openib\_rq\_2\_type}: S
\item {\tt btl\_openib\_rq\_2\_size}: 65536 $->$ 262144 (increment * 2)
\item {\tt btl\_openib\_rq\_2\_num}: 1 $->$ 256 (increment * 2)
\item {\tt btl\_openib\_rq\_2\_low\_wat}: 1 $->$ 64 (increment * 2)
\item {\tt btl\_openib\_rq\_2\_max\_pending\_sends}: 1 $->$ 32 (increment * 2)
\item {\tt btl\_openib\_receive\_queues}: aggregate of the above parameters
\end{itemize}

The following configuration will yield 37730 combinations, after removing
unnecessary combinations that would yield wrong or inaccurate results. The
test was run on the shark cluster at the University of Houston. Shark consists
of 24 dual-core 2.2GHz AMD Opteron nodes connected by a 4xInfiniBand and a
Gigabit Ethernet network interconnect. We used the InfiniBand network
interconnect for our measurements. The total time that the benchmark took was
{\tt 6 hrs 32 min 11 sec}.

\begin{table}[tb]
\centering
\begin{tabular}{|c|c|} \hline
Latency & Number of Combinations \\
\hline
3.87us  & 1\\
\hline
3.88us  & 619\\
\hline
3.89us  & 9702\\
\hline
3.90us  & 10977\\
\hline
3.91us  & 7907\\
\hline
3.92us  & 5076\\
\hline
\end{tabular}  
\caption{OTPO results of the best parameter combinations}
\label{table:results} 
\end{table}

The results that are summarized in table ~\ref{table:results} shows the number
of combinations that are in the range of 0.05 micro-seconds of the best
latency. The best possible combination was with the follwoing values:
$P,131072,64,4:S,262144,256,64,16$
This value for the parameter was the best value between the other values,
however the other values can't be discarded as bad. In the results, it can be
said that most of the parameter combinations that were tested are good
results. 

\subsection{Future Work}
OTPO is a pretty new project. There are lots of enhancements and additional
functionalities that can be added:
\begin{itemize}
\item Adding more benchmarks for the user to choose from (AMB, OSU...)
\item Adding more criteria to base the results on (bandwidth, memory,
  collectives...)
\item Implement a result gathering tool, that takes the results file, and
  presents it to the user in a more readable and interpretable manner.
\item Using methods of sampling to reduce the space of values that need to be
  tested and reduce the number of end results due to the fact that those
  numbers can be huge for any tool to interpret and ofcourse for the user to
  just look at.
\end{itemize}

%%% Local Variables: 
%%% mode: latex
%%% TeX-master: "paper"
%%% End: 
