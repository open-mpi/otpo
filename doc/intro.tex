Open MPI~\cite{gabriel:ompi} is an open source implementation of the
MPI-1~\cite{mpi1} and MPI-2~\cite{mpi2} specifications. The code is
developed and maintained by a consortium consisting of 14
institutions\footnote{As
of January, 2008.} from academia and
industry. The Open MPI design is centered around the Modular Component
Architecture (MCA), which is the software layer providing management
services for Open MPI frameworks. A framework is dedicated to a
single task, such as providing collective operations (i.e., the COLL
framework) or providing data transfer primitives for a particular
network interconnect (i.e., the Byte Transfer Layer framework --
BTL). Each framework will typically have multiple implementations
available in the form of modules (``plugins'') that can be loaded on-demand at run
time.  For example, BTL modules include support for TCP, InfiniBand~\cite{ib},
Myrinet~\cite{myrinet}, shared memory, and others.

Among the management services provided by the MCA is the ability to
accept run-time parameters from higher level abstractions (e.g., {\tt
mpirun}) and pass them down to the according frameworks. MCA runtime
parameters give system administrators, end-users and developers the
possibility to tune the performance of their applications and systems
without having to recompile the MPI library. Examples for MCA runtime
parameters include the values of cross-over points between different 
algorithms in a collective module, or modifying some
network parameters such as internal buffer sizes in a BTL module. Due to its
great flexibility, Open MPI developers made extensively use of MCA runtime parameters.
The current development version of Open MPI has multiple
hundred MCA runtime parameters, depending on the set of modules
compiled for a given platform. While average end-users clearly depend
on developers setting reasonable default values for each parameter,
some end-users and system administrators might explore the
parameter space in order to find values leading to
higher performance for a given application or machine.

In this paper, we introduce OTPO (Open Tool for Parameter
Optimization), a new tool developed in partnership between Cisco
Systems and the University of Houston. OTPO is an Open MPI specific
tool aiming at optimizing the values for a user provided combination
of MCA parameters. Several dependencies might exist between MCA
parameters, some of them even unknown to module developers. OTPO aims
at discovering those hidden dependencies and the effect they have on
performance measured, such as the point-to-point latency or
bandwidth. We present the current status of OTPO and the ongoing work.

The rest of the paper is organized as follows: section~\ref{sec:mot}
presents the parameters of the InfiniBand BTL module (``{\tt openib}'') in order to further motivate the approach taken 
in this project. Section~\ref{sec:impl} discusses some implementation
details of OTPO. In section~\ref{sec:eval}, we present an example for a
parameter optimization using OTPO focusing on MCA parameters from the
{\tt openib} BTL module. Finally, section~\ref{sec:summary} summarizes
the paper and discusses ongoing work.

